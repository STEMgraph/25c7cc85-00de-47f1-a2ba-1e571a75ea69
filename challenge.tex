\begin{challenge}
    \chatitle{Evaluating Complex Expressions with the POSIX Shell}
    \begin{chadescription}
        In the \href{https://github.com/STEMgraph/ac584d6d-1397-49d9-8319-85e9fbd4b765}{Basic Evaluation Challenge} we have learned how to evaluate simple expressions.
        In this challenge we will learn how to use the POSIX Shell to evaluate more complex expressions.
        The POSIX Shell is a standardized command line interface to the operating system.
        What the POSIX standard includes, will be the topic of a future challenge.
        In order to write more complex commands than the previous ones, we will have to wrestle with the concept of identifiers. 
        An identifier is a name that is used to refer to a variable, a function, a type, a value, or something else.
        These identifers can be user-defined or predefined.
        While there are already many predefined identifiers available in your shell-environment, you can also define your own identifiers.
        Defining own identifiers is useful for more complex commands.
        One of the philosophical approaches in software development is called divide-and-conquer.
        This roughly translates to the idea, that a program should be divided into smaller parts.
        These smaller parts should be as simple as possible and independent of each other.
        Every individual part should also be addressable by a well-chosen name, its identifier.
        Identifiers allow us to store parts of our program in memory and to access them later.
        Without going into too much detail ahead of time, we will be working with function and variable identifiers in this challenge.
        In the \href{https://github.com/STEMgraph/ac584d6d-1397-49d9-8319-85e9fbd4b765}{Basic Evaluation Challenge} we used the term expression, without a clear definition, so let's define it now:
        \begin{definition}
            An expression is a combination of operators and operands, that can be reduced to a value. 
            We call this reduction the evaluation of the expression.
            Predefined arrangements of operators and operands, where the operands are replaced with variable values are called functions.
            A variable is a piece of RAM, that is used to store a value and has an identifier.
        \end{definition}

        In this challenge we will: 
        \begin{enumerate}
            \item set identifiers in our shell-environment
            \item we will combine identifiers to form more complex commands
            \item we will investigate the lifetime of identifiers
        \end{enumerate}
    \end{chadescription}

    \begin{task}
        Open your terminal-emulator on your linux system - or any other POSIX system.
        We have worked with expressions already and know, that our shell can evaluate them for us.
        Consecutively we call the system in which we are working now also a command line interpreter.
        Because it interprets our commands, like \textit{"evaluate this expression!"}.
        In addition to the evaluation, our shell can also memorize certain pieces of information.
        The process of making our shell memorize information is called \textit{binding}.
        Binding defines a relationship between an identifier and a piece of information, such as a value or a function.        
        \begin{questions}
            \item Let's bind a value to the identifier \texttt{x}. Type \texttt{x=1} and press \texttt{Enter}. The identifier called \texttt{x} is now related to a piece of RAM, which the value 1 has been assigned to.
            \item Type \texttt{echo \$x} and press \texttt{Enter}. What is the output?
            \item Let's bind a second value to the identifier \texttt{y}. Choose any value you like. Use the \texttt{echo} command to print the value.
            \item We will now evaluate an expression with our identifiers. Type \texttt{expr \$x + \$y} and press \texttt{Enter}. What is the output?
        \end{questions}
    \end{task}

    \begin{task}
        We have now seen how to bind values to identifiers.
        We can also bind functions to identifiers.
        Type the following lines and press \texttt{Enter} after each line:
        \begin{lstlisting}
            function sum() {
            echo $(( $1 + $2 ))
            }
        \end{lstlisting}
        Inside of a shell-function, the identifiers \texttt{1} and \texttt{2} are used to store the values of the function parameters.
        Function parameters are arguments, that are passed to the function when it is called.
        The binding to the identifiers \texttt{1} and \texttt{2} is called binding by position.
        The identifiers are only valid inside of the function, and while the function is executed. 
        For that reason, the identifiers \texttt{1}, \texttt{2} and so on, are called local identifiers. 
        Because they are local to the function, they are not visible outside of the function.
        In contrast to local identifiers, we call identifiers that are visible outside of the function, global identifiers.
        \begin{questions}
            \item In order to see the contents of a function, you can use the \texttt{declare} command. Type \texttt{declare -f sum} and press \texttt{Enter}.
            \item Let's test our function. Type \texttt{sum 1 2} and press \texttt{Enter}. What is the output?
            \item We will now evaluate an expression with our value-identifiers (or variables). Type \texttt{sum \$x \$y} and press \texttt{Enter}. What is the output?
            \item Writing a new function, that uses the same identifier as the first one, will overwrite it.
        \end{questions}
    \end{task}

    \begin{task}
        We can also store the output of an expression behind an identifier.
        The only thing that we need to make sure is, that the expression is evaluated, befor we can store the output in an identifier.
        For this, we will use parentheses, just like in regular math. 
        In addtion to this we want to express to the shell, that it's not just supposed to store the parentheses, but to actually evaluate the expression.
        We use the \texttt{\$}-symbol to express to the command line interpreter, that whatever follows is supposed to be evaluated.
        \begin{questions}
            \item Type \texttt{d=\$(expr 1 + 1)} and press \texttt{Enter}. See the result by typing \texttt{echo \$d}. What is the output?
            \item Type \texttt{e=\$(expr \$x + \$y)} and press \texttt{Enter}. See the result by typing \texttt{echo \$e}. What is the output?
            \item Type \texttt{f=\$(sum \$x \$y)} and press \texttt{Enter}. See the result by typing \texttt{echo \$f}. What is the output?
        \end{questions}
    \end{task}

    \begin{task}
        Function identifiers can also be used in later defined functions.
        Let's define a function first, that prints a string, which is given by a positional parameter.
        \begin{lstlisting}
            function hello() {
            echo "Hello $1"
            }
        \end{lstlisting}
        After writing the closing brace, we get back to the original prompt. 
        Let's define a second function, that calls the first one.
        \begin{lstlisting}
            function world() {
            hello "world"    
            }
        \end{lstlisting}
        
        \begin{questions}
            \item Execute the two functions. Why can the second function call the first one?
            \item Define a third function called \texttt{world2}, that calls \texttt{hello2 "world"}.
            \item Define a fourth function called \texttt{hello2}, that functions exactly like \texttt{hello}. Now \texttt{hello2} was defined after \texttt{world2} was defined. Does this change anything and why?
        \end{questions}
    \end{task}

    \begin{task}
        While we are already in a function, we may also want to create intermediate results.
        For this reason, we can create more identifiers that are local to the function.
        Let's define a function that prints the sum of the square of two numbers.
        \begin{lstlisting}
            function squares() {
            local square1=$(( $1 * $1 ))
            local square2=$(( $2 * $2 ))
            echo $(($square1 + $square2))
            }
        \end{lstlisting}
        \texttt{$(( ))} is a command that is used to evaluate an arithmetic expression, within a function.
        \begin{questions}
            
            
        \end{questions}
    \end{task}
\end{challenge}
    