\learningobjective{At the end of this challenge, the scholar will be able to use the cd-, ls- and pwd-command to traverse their POSIX filesystem.}
\begin{challenge}
    \chatitle{Evaluating Complex Expressions with the Shell}
    \begin{chadescription}
        In the \href{https://github.com/STEMgraph/ac584d6d-1397-49d9-8319-85e9fbd4b765}{Basic Evaluation Challenge} we have learned how to evaluate simple expressions.
        In this challenge we will learn how to use the Shell to evaluate more complex expressions.
        In order to write more complex commands than the previous ones, we will have to wrestle with the concept of identifiers. 
        An identifier is a name that is used to refer to a variable, a function, a type, a value, or something else.
        These identifers can be user-defined or predefined.
        While there are already many predefined identifiers available in your shell-environment, you can also define your own identifiers.
        Defining own identifiers is useful for more complex commands.
        One of the philosophical approaches in software development is called divide-and-conquer.
        This roughly translates to the idea, that a program should be divided into smaller parts.
        These smaller parts should be as simple as possible and independent of each other.
        Every individual part should also be addressable by a well-chosen name, its identifier.
        Identifiers allow us to store parts of our program in memory and to access them later.
        Without going into too much detail ahead of time, we will be working with function and variable identifiers in this challenge.
        In the \href{https://github.com/STEMgraph/ac584d6d-1397-49d9-8319-85e9fbd4b765}{Basic Evaluation Challenge} we used the term expression, without a clear definition, so let's define it now:
        \begin{definition}
            An expression is a combination of operators and operands, that can be reduced to a value. 
            We call this reduction the evaluation of the expression.
            Predefined arrangements of operators and operands, where the operands are replaced with variable values are called functions.
            A variable is a piece of RAM, that is used to store a value and has an identifier.
        \end{definition}

        In this challenge we will: 
        \begin{enumerate}
            \item set identifiers in our shell-environment
            \item we will combine identifiers to form more complex commands
            \item we will investigate the lifetime of identifiers
        \end{enumerate}
    \end{chadescription}

    \begin{task}
        Open your terminal-emulator on your linux system - or any other *NIX system.
        We have worked with expressions already and know, that our shell can evaluate them for us.
        Consecutively we call the system in which we are working now also a command line interpreter.
        Because it interprets our commands, like \textit{"evaluate this expression!"}.
        In addition to the evaluation, our shell can also memorize certain pieces of information.
        The process of making our shell memorize information is called \textit{binding}.
        Binding defines a relationship between an identifier and a piece of information, such as a value or a function.        
        \begin{questions}
            \item Let's bind a value to the identifier \texttt{x}. Type \texttt{x=1} and press \texttt{Enter}. The identifier called \texttt{x} is now related to a piece of RAM, which the value 1 has been assigned to.
            \item Type \texttt{echo \$x} and press \texttt{Enter}. What is the output?
            \item Let's bind a second value to the identifier \texttt{y}. Choose any value you like. Use the \texttt{echo} command to print the value.
            \item We will now evaluate an expression with our identifiers. Type \texttt{expr \$x + \$y} and press \texttt{Enter}. What is the output?
        \end{questions}
    \end{task}

    \begin{task}
        We have now seen how to bind values to identifiers.
        We can also bind functions to identifiers.
        \begin{lstlisting}
            function say_hello(){ echo "Hello" }
        \end{lstlisting}
        We can now call the function by typing \texttt{say_hello} and pressing \texttt{Enter}.
        In order to see the contents of a function, you can use the \texttt{declare} command. Type \texttt{declare -f say_hello} and press \texttt{Enter}.
        We can also define functions that have multiple lines. 
        Just write an opening curly bracket \texttt{\{} and press \texttt{Enter}.
        The command-line interpreter will then continue to let you input the lines until you set a closing curly bracket \texttt{\}}.
        Try typing in the following example:
        \begin{lstlisting}
            function say_multiple_things(){
            echo "Hello"
            echo "I"
            echo "say"
            echo "multiple"
            echo "things"
            }
        \end{lstlisting}
    \end{task}

    \begin{task}
        A function that always returns the exact same value in not very useful.
        That's why we can also bind arguments to identifiers that can be used inside of the function.
        Let's see, whether we can make the behavior of a function more useful, by depending on the arguments that are passed to it.
        Think of a mathematical expression, that depends on a variable \texttt{x}.
        Let's say, that you want to calculate how much a certain number \texttt{n} of apples costs.
        Mathematically you would write the following expression:
        \( costs (n) = n \times 0.2 \)
        The expression \texttt{n} is the argument of that expression.
        Let's write the function in a more programming-like way.
        \begin{lstlisting}
            function: costs( number of apples ) {
                cost of one apple = 0.2
                print: number of apples * cost of one apple
            }
        \end{lstlisting}
        What you see there is so called \textit{pseudo code}.
        Pseudo code helps us expressing processes in natural language, but still in a programming-like way.
        We give the function a clear identifier, so we can call it later. 
        Just as a mathematical function, our programming function has an argument that it excepts.
        Within the curly brackets, we write what should be calculated, every time the function is called.
        In this block, we can now write down one statement per line, which will be executed in the exact order.
        We can define helper identifers and assign values to them.
        Our function ends, when the closing curly bracket is set.
        Just to be clear, this is not the same as defining a function in a programming language, it is meant to be a template that can be translated into any programming language. 
        Since we are using the \textit{bash} shell, let's translate our template into \textit{bash} code:
        \begin{lstlisting}
            function costs () {
                cost_of_one_apple=0.2
                echo $(($1 * $cost_of_one_apple))
            }   
        \end{lstlisting}
        A few things changed now:
        \begin{itemize}
            \item There are no longer arguments in the parentheses, but is accessed by the identifier \texttt{1}
            \item \texttt{\$}-signs appear in the function body
        \end{itemize}
        There are multiple different situations in which we need the \texttt{\$} to communicate to the script-interpreter that we need it to do something special. 
        In the case displayed here, therer are two ways the \texttt{\$} is used:
        \begin{itemize}
            \item Accessing an argument that was bound to the identifier \texttt{\$1} 
            \item Evaluating an arithmetical expression
        \end{itemize}
        In \texttt{bash}-functions, we don't use variable-names in the functions arguments, but we address the arguments positionally. 
        Imagine you call a function from your command-line like:
        \begin{lstlisting}
            user@host ~> echo "Hello World"
        \end{lstlisting}
        Since the command-line interpreter is a function running on your computer itself, it sees both of the strings given - \texttt{echo} and \texttt{"Hello World"} - as arguments to its own function. 
        Within the command-line interpreter, these arguments are only refered to as \textit{the first parameter given}, \textit{the second argument given} and so on and so forth. 
        Counting in binary usually starts with zero, therefor the argument \texttt{echo} is actually the \textit{zeroth} argument. 
        The command-line interpreter uses the zeroth argument as an identifier for what to do next and the first argument as an argument to pass to that function. 
        If more arguments follow, these would be passed as well and could be accessed by the identifiers \texttt{\$2}, \texttt{\$3}, \texttt{\$4} and so on.
        These identifiers are only accessible within the function, that's why we say they belong to a category called \textit{local identifiers}.
        \begin{questions}
            \item Let's test our function. Type \texttt{costs 1} and press \texttt{Enter}. What is the output?
            \item Change the function, that it accesses two arguments. Make the price of one apple a variable, depending on the second argument. What does the code look like?
            \item We will now evaluate an expression with our shell value-identifiers (or variables) that we already defined earlier. Type \texttt{costs \$x \$y} and press \texttt{Enter}. What is the output?
        \end{questions}
    \end{task}

    \begin{task}
        We can also memorize the output of an expression behind an identifier.
        The only thing that we need to make sure is, that the expression is evaluated, befor we store the output in an identifier.
        For this, we will use parentheses, just like in regular math. 
        In addtion to this we want to express to the shell, that it's not just supposed to store the parentheses, but to actually evaluate the expression.
        We use the \texttt{\$}-symbol to express to the command line interpreter, that whatever follows is supposed to be evaluated.
        \begin{questions}
            \item Type \texttt{d=\$(expr 1 + 1)} and press \texttt{Enter}. See the result by typing \texttt{echo \$d}. What is the output?
            \item Type \texttt{e=\$(expr \$x + \$y)} and press \texttt{Enter}. See the result by typing \texttt{echo \$e}. What is the output?
            \item Type \texttt{apple_price_tag=\$(sum \$x \$y)} and press \texttt{Enter}. See the result by typing \texttt{echo \$f}. What is the output?
        \end{questions}
    \end{task}

    \begin{task}
        ALready in a Shell Session defined function-identifiers can also be used in other functions.
        Let's define a function first, that prints a string, which is given by a positional parameter.
        \begin{lstlisting}
            function my_hello_function () { 
            echo "Hello $1" 
            }
        \end{lstlisting}
        After writing the closing brace, we get back to the original prompt. 
        Let's define a second function, that calls the first one.
        \begin{lstlisting}
            function call_other_func() { 
            my_hello_function "from another function" 
            }
        \end{lstlisting}
        
        \begin{questions}
            \item Execute the two functions. Why can the second function call the first one?
            \item Define a third function called \texttt{call_other_func2}, that calls \texttt{my_hello_function2 "world"}.
            \item Define a fourth function called \texttt{my_hello_function2}, that functions exactly like \texttt{my_hello_function}. Now \texttt{my_hello_function2} was defined after \texttt{call_other_func2} was defined. Does this change anything and why?
        \end{questions}
    \end{task}

    \begin{task}
        While we are already in a function, we may also want to create intermediate results.
        For this reason, we can create more identifiers that are local to the function.
        Let's define a function that prints the sum of the square of two numbers.
        \texttt{\$(( ))} is a command that is used to evaluate an arithmetic expression, within a function.
        \begin{lstlisting}
            function squares() {
            local square1=$(( $1 * $1 ))
            local square2=$(( $2 * $2 ))
            echo $(($square1 + $square2))
            }
        \end{lstlisting}
        \begin{questions}
            \item Run \texttt{echo \$square1}. What is the output and why?
            \item Overwrite the function, but without the \texttt{local}-keyword. What happens?
            \item Write two new functions, called \texttt{sum_of_squares} and \texttt{product_of_squares}. Both of them should take two arguments and return the sum and the product of the square of the arguments, respectively. Continue writing a third function called \texttt{compare_squares}. This function takes two arguments, calles the first two functions and echos a \texttt{1} if the sum is greater and a 0 if the product is greater.
        \end{questions}
    \end{task}
\end{challenge}
    
